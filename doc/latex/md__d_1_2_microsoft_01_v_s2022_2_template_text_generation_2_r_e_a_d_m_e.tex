\chapter{Template\+Text\+Generation}
\hypertarget{md__d_1_2_microsoft_01_v_s2022_2_template_text_generation_2_r_e_a_d_m_e}{}\label{md__d_1_2_microsoft_01_v_s2022_2_template_text_generation_2_r_e_a_d_m_e}\index{TemplateTextGeneration@{TemplateTextGeneration}}
\label{md__d_1_2_microsoft_01_v_s2022_2_template_text_generation_2_r_e_a_d_m_e_autotoc_md0}%
\Hypertarget{md__d_1_2_microsoft_01_v_s2022_2_template_text_generation_2_r_e_a_d_m_e_autotoc_md0}%
Программа генерирует текст по шаблону. Запускается через консоль и принимает 3 параметра. Пример запуска из командной строки\+: Template\+Text\+Generation.\+exe ./template.txt ./input.csv ./output.txt Входные данные представляются в виде трех файлов\+: 1) Шаблон письма с расширением .txt. Шаблон имеет метки для подстановки. Метки для подстановки обозначаются текстом между двух символов \"{}\#\"{} 2) Входные данные для подстановки с расширением .csv. Данные для подстановки имеют метки и значения, которыми нужно их заменить. Данные для подстановки записываются в одну ячейку. Данные для подстановки можно записывать только в первой колонке файла. Данные для подстановки записываются в формате \"{}\#\+TAG\#ЗНАЧЕНИЕ\"{}. Данные для подстановки записываются в разных строках. Значением для подстановки может быть многострочный текст. Если значений для подстановки на одну метку несколько, то метку заменит первое по порядку записанное значение в файле для этой метки. 3)Выходной файл должен быть пустым с расширением .txt. Выходные данные\+: Выходной файл -\/ он должен содержать сгенерированный текст с подставленными данными на места для подстановки. 